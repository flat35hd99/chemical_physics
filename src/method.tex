前回の復習

(T,V) \rightarrow (T',V')に対して、WとQの差

\begin{equation}
  Q[(T,V) \rightarrow (T',V')] - W[(T,V) \rightarrow (T',V')]
\end{equation}

は、

1. 過程によらず
2. 始状態と終状態で定まる

1.3.3

\subsection*{内部エネルギーの導出}

既知としている量から内部エネルギー$U$を導きたい。そのために、$U$を状態方程式、熱容量から求める。

\begin{enumerate}
  \item T一定でのU(T,V)のT依存性(法則1)から、
  \begin{eqnarray}
    U(T_1, V) - U(T_0, V) &=& Q[(T_0,V) \rightarrow (T_1,V)]
                          &=& \int_{T_0}^{T_1} C_V (T,V) dT
    \label{hogehoge}
  \end{eqnarray}
  $C_V (T,V) > 0$とEq.\ref{hogehoge}より、定理1.1($U vs T$) $T_1 > T_0$に対して$U(T_0, V) < U(T_1, V)$で、Uは単調増加関数であることがわかる。また、
  \begin{equation}
    \left(\frac{dU(T:V)}{dT}\right)_V = C_V (T:V)
  \end{equation}
  \item T一定におけるU(T:V)のV依存性
  \begin{equation}
    \left(\frac{dU(T:V)}{dV}\right)_T = -P + T \frac{dP}{dT}_V
  \end{equation}
\end{enumerate}

\begin{equation}
  F &=& U - TS
\end{equation}

より

\begin{eqnarray}
  \left(\frac{\partial F}{\partial V}\right)_T = \left(\frac{\partial U}{\partial V}\right)_T - T\left(\frac{\partial S}{\partial V}\right)_T
\end{eqnarray}

$S = - \frac{\partial F}{\partial T}$なので、

\begin{eqnarray}
  \left(\frac{\partial S}{\partial V}\right) &=& - \frac{\partial}{\partial V}\left(\frac{\partial F}{\partial T}\right)
    &=& - \frac{\partial}{\partial T}\left(\frac{\partial F}{\partial V}\right)
    &=& \frac{\partial P}{\partial T}
\end{eqnarray}

また、

\begin{equation}
  \left(\frac{\partial U}{\partial V}\right)_T = -P + T \frac{\partial P}{\partial T}_V
\end{equation}

であるから、よって Eq.\ref{hogehoge}は

\begin{eqnarray}
  U(T_1, V) - U(T_0, V) 
     &=& \int_{V_0}^{V_1} \left(\frac{\partial U}{\partial V}\right)_T dV
  \label{hogehoge}
\end{eqnarray}

となる。例えば、理想気体の場合、$\left(\frac{\partial U}{\partial V}\right) = 0$になるため、$U(T:V)$は$V$に依存しない。

\begin{equation}
  U(T:V) = CNRT + U_*
\end{equation}

ただし$U_*$は資料的な定数である。

\subsection*{熱力学第二法則}

\begin{enumerate}
  \item 熱力学第一法則はWとQの透過性を論じたのに対し
  \item 熱力学第二法則はWとQの相違を述べる(熱と仕事は違う)
\end{enumerate}

前提1.8 ケルビンの原理

流体がサイクル過程で(始状態と終状態が同じ過程)ある熱浴から$Q>0$を奪い、他に変化を残すことなく外界に吸収した熱量$Q$と同じ大きさの仕事$W>0$をすることはできない。

\subsubsection{等温過程でのケルビンの原理}

Note: 等温過程は\textbf{1つの}熱浴との間の熱のやり取りと熱力学的操作からなる。

Note: 等温 is isothermal \rightarrow i過程

前提1.8' : i過程におけるケルビンの原理